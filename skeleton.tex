\documentclass{scrartcl}
\KOMAoptions{pagesize,paper=landscape,DIV=20}
\usepackage{amssymb}
\usepackage{array}
\usepackage{ngerman}
\usepackage[utf8]{inputenc}
\title{RCPSP-OC}
\date{}
\begin{document}
\setlength\extrarowheight{5pt}
\section{Ergebnisse}
%%CONTENTS%%
\section{Legende}
\begin{itemize}
\item $(\lambda|z_r)$ ($(\lambda|z_{rt})$): Möglichst nutzbare Zusatzkapazität je Ressource (und Periode) in GA bestimmen
\item $(\lambda|\beta)$: je AG über erlaubte ZK-Nutzung entscheiden mit $|\{l,L\} \times \{s, S\} \times \{u, U\}|=8$ Varianten
	\begin{itemize}
	\item $l$: Eintrag in $\beta$-Vektor bezieht sich auf Arbeitsgang
	\item $L$: Eintrag in $\beta$-Vektor bezieht sich auf Listenposition
	\item $s$: Gemeinsamer One Point Crossover mit einer ``Zerhackstelle'' $q$
	\item $S$: Zwei getrennte One Point Crossovers mit unabhängigen ``Zerhackstellen'' $q_1, q_2$
	\item $u$: Serielles SGS ``von unten'' ($\beta=1$ AG nutzt zuerst Normalkapazität dann ZK)
	\item $U$: Serielles SGS ``von oben'' ($\beta=1$ AG nutzt zuerst ZK dann Normalkapazität)
	\end{itemize}
\item $(\lambda|\tau)$ mit $\tau \in [0,1]$: In Zeitfenster zwischen Reihenfolge- und Ressourcenzulässigkeit Wahl per GA treffen
\item $(\lambda)$: Einplanung in Zeitfenster durch heuristisches ``Ausprobieren'' (Vervollständigung per seriellem SGS ohne ZK)
\end{itemize}
\end{document}
